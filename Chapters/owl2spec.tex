\chapter{Chi tiết Ontology Web Language}
\paragraph{Giới thiệu } - Như đã được đề cập trong phần cuối của chương trước, chức năng chính của OWL là một ngôn ngữ ontology cung cấp ngữ nghĩa cho Semantic Web. Trong nội dung chương này, chúng em sẽ giới thiệu về cú pháp, định dạng và chi tiết các đặc tính của ngôn ngữ Ontology Web. Phiên bản Ontology Web Language chúng em sử dụng là phiên bản 2 được tổ chức W3C khuyến khích sử dụng so với phiên bản OWL 1.1 .
\section{Khái quát về OWL 2 \cite{owl2}}
\subsection{Tổng quan}
\begin{figure}[ht!]
	\centering
	\includegraphics[width=120mm]{Figures/owl2structure.png}
	\caption{Cấu trúc của OWL 2\label{overflow}}
\end{figure}
Hình trên cho chúng ta cái nhìn tổng quan về các định dạng file, các loại cú pháp và cách khả năng serialization thành RDF Graph của Ontology. Như chúng ta thấy trong hình thì hình eclipse ở giữa thể hiện khái niệm trừu tượng của một ontology, có thể hiểu là một cấu trúc trừu tượng hay một đồ thi RDF. Chúng ta có thể dùng nhiều cú pháp để biểu diễn ontology và định dạng chúng dưới dạng file khác nhau (Syntex layer trong hình), các định dạng và cú pháp này hoàn toàn có thể chuyển đổi qua lại với nhau. Lớp ngữ nghĩa trong hình (semantic layer) cho thấy ngữ nghĩa được quy định theo 2 tiêu chuẩn kỹ thuật khác nhau là Direct Semantics và RDF-Based Semantics.
\\
Phần lớn những người phát triển Ontology bằng OWL2 sẽ chỉ cần 1 cú pháp (tương đương với 1 định dạng file) và một dạng biểu diễn ngữ nghĩa.
\subsection{Ontologies}
Bất kì ontology OWL2 nào đều có thể được định dạng như một đồ thị RDF. Mối quan hệ giữa 2 cách này được quy định bới cách tài liệu Mapping to RDF Graphs document [\href{http://www.w3.org/TR/owl2-overview/#ref-owl-2-rdf-mapping}{OWL 2 RDF Mapping}] \cite{mapping_rdf_graph}, trong tài liệu này định nghĩa rất rõ ràng một bảng map từ định dạng cấu trúc của ontology qua đồ thị RDF, và ngược lại. 
\subsection{Cú pháp}
Trong thực tế, một cú pháp cụ thể rất cần thiết để lưu trữ các OWL2 Ontologies và để trao đổi chúng giữa các công cụ và ứng dụng khác nhau. Cú pháp đầu tiên có khả năng hoán đổi là RDF/XML [\href{http://www.w3.org/TR/owl2-overview/#ref-rdf-syntax}{RDF Syntax}] \cite{rdfxml}. Ngoài RDF/XML có khả năng cung cấp khả năng tương tác giữ nhiều ứng dụng OWl2 khác nhau, các loại cú pháp khác đều có thể được sử dụng. Dưới đây là bảng so sánh và liệt kê các cú pháp.
\begin{table}[h]
\begin{tabular}{ |l|l|l|p{4cm}|}

\hline
Tên cú pháp & Mô tả & Trạng thái & Mục đích sử dụng\\
\hline
RDF/XML & Mapping to RDF Graphs \cite{mapping_rdf_graph} \cite{rdfxml} & Bắt buộc & Hoán đổi được ( có thể viết và đọc được bằng nhiều phần mềm OWL2)
\\
\end{tabular}
\caption{Bảng so sánh các cú pháp của OWL2\label{overflow}}
\end{table}
\section{Các đặc tính chi tiết của OWL2}
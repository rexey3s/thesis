\chapter{Kết luận}
Với sự phát triển ngày càng nhanh chóng và mạnh mẽ của các mô hình dữ liệu mới mà đặc biệt là web ngữ nghĩa (Semantic Web) nên trong tương lai gần các hệ thống dữ liệu sẽ sử dụng những quy ước của Semantic Web sẽ ngày càng tăng lê. Đây là một điều dễ hiểu vì những lợi ích mà nó mang lại so với cách tổ chức dữ liệu truyền thống (Open World vs. Closed World Assumption). Điều đó cũng đồng nghĩa là sẽ có nhiều dữ liệu sẽ được tổ chức dưới dạng Ontology Web Language hay dưới các phiên bản RDFS nên việc chúng em tin rằng việc xây dựng được một ứng dụng web phục vụ cho việc phát triển và biên tập ontology sẽ hỗ trợ rất nhiều cho các nhà phát triển mô hình dữ liệu dựa trên OWL 2. Bên cạnh đó, tuy còn ontology phân loại hàng hóa của chúng em còn khá sơ khai nhưng nó cũng chứng minh được thêm một khả năng thực tiễn mà ontology đem lại chính là phân loại các cá thể hàng hóa.
\section{Những công việc đã làm được}
\begin{itemize}
\item Có được những kiến thức nền tảng chắc chắn về ngôn ngữ Ontology Web Language 2, Semantic Web Rule Language và đặc tính suy luận của chúng.
\item Biết được tính nhất quán của ontology  và các nguyên nhân phổ biến gây ra tính thiếu nhất quán trong ontology.
\item Xây dựng được một ứng dụng dùng để thiết kế ontology trên môi trường Web với giao diện thân thiện, dễ dùng và có đầy đủ các tính năng như suy luận, giải thích hỗ trợ biên tập SWRL Rule.
\item Thiết kế môt OWL2 ontology để trình bày tính năng phân loại và tính năng hỗ trợ phân loại từ ứng dụng đã thiết kế.
\end{itemize}
Tóm lại, đề tài nghiên cứu và "xây dựng ontology phục vụ cho việc phân loại hàng hóa tự động" đã cho thấy tính khả thi trong việc kết hợp ngôn ngữ Ontology Web Language 2 (OWL2), ngôn ngữ SWRL Rule và cách thức hoạt động của reasoner để phân loại các cá thể một cách tự động. Bên cạnh đó, công cụ chỉnh sửa ontology OWL Editor mà nhóm phát triển đem lại nhiều lợi ích trong việc nghiên cứu và phát triển ontology, cụ thể hơn là đơn giản hoá việc thiết kế một ontology theo ý muốn. Sau khi đạt được những thành công kể trên, đề tài còn có thể đạt được những thành công thực tiễn hơn nếu được áp dụng vào các lĩnh vực khác, chẳng hạn như việc phân loại hàng hoá, hoặc phân loại các gói tin trong lưu lượng mạng... 
\section{Những mặt hạn chế}
\begin{itemize}
\item Trình biên tập còn nhiều điểm phản hồi chưa tốt do chúng em chưa chỉ tập trung vào tính năng , vẫn chưa tính đến hiệu năng cũng như hiệu suất của ứng dụng , như việc phải chịu nhiều tải cùng lúc, khả năng lưu trữ, quản lý người dùng.
\item Trình biên tập chưa có cơ chế đăng nhập hoặc quản lý các tài liệu ontology.
\item Ý tưởng về việc phân loại còn chỉ ở dạng mô hình, chưa có tính thực nghiệm cao.
\end{itemize}
Sau đây là bảng so sánh những tính năng giữa ứng dụng phát triển OWL-Editor của chúng em và Protege - ứng dụng tương tự cũng dùng để phát triển và xây dựng ontology được phát triển bởi nhóm tác giả của Đại Học Stanford và đã được cộng đồng người dùng ontology sử dụng rất phổ biến.

\begin{table}[h!]
	\centering
	\begin{tabular}{|p{4cm}|p{5cm}|p{5cm}|}
		\hline
						& OWL-Editor & Protege Desktop \\\hline
		Môi trường		& Web (không cần cài đặt) & Desktop (Yêu cầu cài đặt) \\\hline	
		Nền tảng		& Spring Boot và Vaadin Framework & Java Swing, OSGi Framework \\\hline
		Phiên bản OWL-API & 4.0.0 &  3.5.1 \\\hline
		Reasoner hỗ trợ & Tích hợp Pellet Reasoner & Pellet, HermitT, Fact++ (Phải cài thêm plugin)\\\hline
		Hỗ trợ import ontology khác vào ontology hiện hành & Chưa có & Đã hỗ trợ \\\hline
		Hỗ trợ plugins & Chưa có & Plugin phong phú do cộng đồng phát triển \\\hline
		Hỗ trợ biên tập SWRL Rule & Sử dụng SWRL-API với đầy đủ các built-in & Bản 4.3 sử dụng chuẩn SWRL cũ nên không khai thác được nhiều built-in, bản 5.0-beta có hỗ trợ nhưng đòi hỏi thao tác cài đặt khả rườm rà \\\hline
		Hỗ trợ biên tập Class Expression & Tính năng autocomplete còn đơn giản & Tính autocomplete có khả năng dự đoán được các từ vựng kế tiếp \\\hline
		Miêu tả các SWRL Rule qua biểu đồ& Vẽ biểu đồ giúp miêu tả các điều kiện-kết quả của tậo những rule có cùng một lớp trong điều kiện & Không có tính năng này \\\hline
		Tính năng gợi ý & Đưa ra các câu hỏi gợi ý về nhằm giúp người phát triển đơn giản hóa việc thêm các thuộc tính cho cá thể & Không có tính năng này \\\hline
		Vẽ biểu đồ lớp và cá thể & Vễ một sơ đồ phân lớp và cá thể & Yêu cầu phải cài thêm plugin \\\hline
		\hline
	\end{tabular}
	\caption{Bảng so sánh tính năng của OWL-Editor và Protege \label{overflow}}  
\end{table}

\section{Hướng phát triển}
Nhẳm khắc phục một số hạn chế kể trên chúng em cũng đã nghĩ đến các hướng phát triển sau :
\begin{itemize}
\item Để tăng tốc độ xử lý, chúng em sẽ sử dụng kết hơp Vaadin với Spring Boot (đã thực hiện được một phần), đồng thời tố ưu hóa các đoạn code.
\item Trong giai đoạn phát triển cuối cùng, chúng em đã tìm hiểu được là đã có những cơ sở dữ liệu hỗ trợ lưu tài liệu Ontology dưới dạng RDF Graph, tiêu biểu có Stardog - một cơ sở dữ liệu Semantic hỗ trợ OWL 2, SWRL, và đặc biệt tích hợp bản mới nhất của reasoner Pellet 3.0. Sẽ là một ý tưởng tuyệt vời nếu tích hợp sử dụng Stardog là nơi lưu trữ những tài liệu ontology mà người dùng soạn thảo, và giảm tải khả năng suy luận cho cơ sở dữ liệu,
\item Thiết kế một ontology với số lượng lớn các phát biểu phục vụ cho việc phân loại trên thực tế, lĩnh vực là rất rộng lớn nhưng chúng em đã nghĩ đến một lĩnh vực thực tiễn và có thể sử dụng các kiến thức chuyên ngành mạng, đó là phân loại thông tin lưu lượng mạng nhằm phục vụ cho các tác vụ như xây dựng Firewall rule, IDS rule.
\item Đưa ứng dụng thành một ứng dụng mã nguồn mở.
\end{itemize}

Đề tài mà chúng em đã thực hiện, với tính năng phân loại tự động đắt giá, có thể được áp dụng vào nhiều lĩnh vực thực tế khác nhau. Dưới đây là một số ứng dụng có tính khả thi cao :
\begin{itemize}
\item Phân loại các hàng hoá dựa theo các đặc điểm, tính năng đặc trưng của nó
\item Phân loại thông tin lưu lượng mạng nhằm phục vụ cho việc phân tích, ngăn chặn các cuộc tấn công an ninh mạng
\end{itemize}

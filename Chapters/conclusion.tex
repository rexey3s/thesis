\chapter{Kết luận}
Với sự phát triển ngày càng nhanh chóng và mạnh mẽ của công nghệ Web 3.0 hay Semantic Web, trong tương lai các hệ thống dữ liệu sẽ sử dụng những nguyên lý của Semantic Web sẽ ngày càng tăng lên do những lợi ích mà nó mang lại so với cách tổ chức dữ liệu truyền thống (Open World vs. Closed World Assumption). Điều đó đồng nghĩa với việc nhu cầu về thiết kế các mô hình dữ liệu Ontology cũng sẽ tăng theo. Bên cạnh với những lý thuyết về việc sử dụng Ontology cho việc phân loại đã được thực nghiệm trong đề tài, chúng em tin rằng khả năng phát triển một hệ thống thực tế với một ontology với lượng phát biểu lớn về dữ liệu và thông tin của chúng là khả thi và có thể thực hiện bằng các công nghệ hiện tại.
\section{Những việc đã làm được}
\begin{itemize}
\item Nắm được những kiến thức nền tảng nhất của ngôn ngữ OWL 2, SWRL.
\item Tìm hiểu được các cơ chế, giải thuật xây dựng các giải thích cho phát biểu được suy luận ra từ Ontology.
\item Xây dựng được một ứng dụng biên tập ontology với giao diện thân thiện, dễ dùng và có đầy đủ các tính năng như suy luận, giải thích hỗ trợ biên tập SWRL Rule.
\item Xây dựng môt OWL2 ontology phân loại và tính năng hỗ trợ phân loại từ SWRL Rule.
\end{itemize}
Tóm lại, đề tài "Nghiên cứu và phát triển hệ thống phân loại tự động" đã cho thấy tính khả thi trong việc kết hợp ngôn ngữ Ontology Web Language 2 (OWL2), ngôn ngữ SWRL Rule và cách thức hoạt động của reasoner để phân loại các cá thể một cách tự động. Bên cạnh đó, công cụ chỉnh sửa ontology UIT-OWL Editor mà nhóm phát triển đem lại nhiều lợi ích trong việc nghiên cứu và phát triển ontology, cụ thể hơn là đơn giản hoá việc thiết kế một ontology theo ý muốn. Sau khi đạt được những thành công kể trên, đề tài còn có thể đạt được những thành công thực tiễn hơn nếu được áp dụng vào các lĩnh vực khác, chẳng hạn như việc phân loại hàng hoá, hoặc phân loại các gói tin trong lưu lượng mạng...
\section{Những mặt hạn chế}
\begin{itemize}
\item Trình biên tập còn nhiều điểm phản hồi chưa tốt do chúng em chưa chỉ tập trung vào tính năng , vẫn chưa tính đến hiệu năng cũng như hiệu suất của ứng dụng , như việc phải chịu nhiều tải cùng lúc, khả năng lưu trữ, quản lý người dùng ...
\item Trình biên tập chưa có cơ chế đăng nhập hoặc quản lý các tài liệu ontology.
\item Ý tưởng về việc phân loại còn chỉ ở dạng mô hình, chưa có tính thực nghiệm cao.
\end{itemize}
\section{Hướng phát triển}
Nhẳm khắc phục một số hạn chế kể trên chúng em cũng đã nghĩ đến các hướng phát triển sau :
\begin{itemize}
\item Để tăng tốc độ xử lý, chúng em sẽ sử dụng kết hơp Vaadin với Spring Boot (đã thực hiện được một phần), đồng thời tố ưu hóa các đoạn code.
\item Trong giai đoạn phát triển cuối cùng, chúng em đã tìm hiểu được là đã có những cơ sở dữ liệu hỗ trợ lưu tài liệu Ontology dưới dạng RDF Graph, tiêu biểu có Stardog - một cơ sở dữ liệu Semantic hỗ trợ OWL 2, SWRL, và đặc biệt tích hợp bản mới nhất của reasoner Pellet 3.0. Sẽ là một ý tưởng tuyệt vời nếu tích hợp sử dụng Stardog là nơi lưu trữ những tài liệu ontology mà người dùng soạn thảo, và giảm tải khả năng suy luận cho cơ sở dữ liệu,
\item Thiết kế một ontology với số lượng lớn các phát biểu phục vụ cho việc phân loại trên thực tế, lĩnh vực là rất rộng lớn nhưng chúng em đã nghĩ đến một lĩnh vực thực tiễn và có thể sử dụng các kiến thức chuyên ngành mạng, đó là phân loại thông tin lưu lượng mạng nhằm phục vụ cho các tác vụ như xây dựng Firewall rule, IDS rule.
\item Đưa ứng dụng thành một ứng dụng mã nguồn mở.
\end{itemize}

Đề tài mà chúng em đã thực hiện, với tính năng phân loại tự động đắt giá, có thể được áp dụng vào nhiều lĩnh vực thực tế khác nhau. Dưới đây là một số ứng dụng có tính khả thi cao :
\begin{itemize}
\item Phân loại các hàng hoá dựa theo các đặc điểm, tính năng đặc trưng của nó
\item Phân loại thông tin lưu lượng mạng nhằm phục vụ cho việc phân tích, ngăn chặn các cuộc tấn công an ninh mạng
\end{itemize}

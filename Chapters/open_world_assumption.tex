\chapter{Giới thiệu về Semantic Web và Open World Assumption}
Trước khi bắt đầu giới thiệu với sâu hơn về Ontology Web Language (OWL), chúng em xin được giới thiệu qua về giả định Thế Giới Mở (Open World Assumption - OWA) được Semantic Web chấp nhận và phân biệt giả định này với giả định Thế Giới Đóng (Closed World Assumption - CWA).
\begin{description}
\item[Closed World Assumption] 
Giả định Thế Giới Đóng (CWA) là giả định mà những điều không chắc hoặc không có cơ sở để chứng minh là \textbf{đúng} sẽ được chấp nhận là \textbf{sai}.
\item[Open World Assumption]
Giả định Thế Giới Mở (OWA) thì ngược lại, với những điều không chắc hoặc không có cơ sở để chứng minh là \textbf{đúng} sẽ được chấp nhận là \textbf{chưa biết}. 
\item[Ví dụ]
Xem xét một câu nói sau đây: "A là một công dân của nước Mỹ". Nếu có ai đó hỏi "A có phải là một công dân của Việt Nam hay không ?". Xét theo CWA, câu trả lời là \textit{không}, ngược lại với OWA thì câu trả lời là \textit{chưa biết}. 
\end{description}
\section{Vậy OWA và CWA được sử dụng khi nào ?}
Giả định thế giới đóng (CWA) được sử dụng khi một hệ thống đã có đầy đủ thông tin. Đây là trường hợp được áp dụng cho nhiều ứng dụng cơ sở dữ liệu. Ví dụ, xem xét một tình huống một ứng dụng cơ sở dữ liệu đặt vé máy bay, chúng ta tìm kiếm đường bay thẳng Phú Quốc và Hà Nội, và kết quả là nó không tồn tại trong cơ sở dữ liệu (không quan tâm đến thực tế có hay không có đường bay này). Và theo CWA nên câu trả lời từ cơ sở dữ liệu là : "Không có đường bay thẳng Hà Nội - Phú Quốc" (Một giả định là thực tế cũng không tồn tại đường bay này do cơ sở dữ liệu không biết). Đây là dạng ứng dụng mà người dùng mong đợi một câu trả lời chính xác ( phổ biến ở các cở sở dữ liệu quan hệ).
Ngược lại với Giả định thế giới đóng, Giả định thế giới mở đươc áp dụng trên một hệ thống mà thông tin được cung cấp không đầy đủ. Đây là trường hợp chúng ta một biểu dạng một dạng tri thức (a.k.a Ontologies) và chúng ta muống khám phá những thông tin mới tiềm ẩn trong đó. Ví dụ, xem xét một hệ thống lưu trữ tiền sử bệnh lý của bệnh nhân. Nếu cơ sở dữ liệu không chứa thông tin về một dạng dị ứng cụ thể mà bệnh nhân mắc phải, điều đó không đồng nghĩa là bệnh nhân đó không mắc phải nó trên thực tế. Từ đó câu trả lời từ cở sở dữ liệu theo chuẩn OWA sẽ là : "Không rõ bệnh nhân này có mắc phải dị ứng đó không, trừ khi những thông tin đầy đủ hơn được cung cấp".
\section{CWA vs. OWA: Ví dụ}
Giả định Thế Giới Đóng không chỉ là trả về các câu trả lời \textit{"không"} 
\chapter{Giới thiệu về Semantic Web và Open World Assumption}
Trước khi bắt đầu giới thiệu với sâu hơn về Ontology Web Language (OWL), chúng em xin được giới thiệu qua về giả định Thế Giới Mở (Open World Assumption - OWA) được Semantic Web chấp nhận và phân biệt giả định này với giả định Thế Giới Đóng (Closed World Assumption - CWA).
\begin{description}
\item[Closed World Assumption] 
Giả định Thế Giới Đóng (CWA) là giả định mà những điều không chắc hoặc không có cơ sở để chứng minh là \textbf{đúng} sẽ được chấp nhận là \textbf{sai}.
\item[Open World Assumption]
Giả định Thế Giới Mở (OWA) thì ngược lại, với những điều không chắc hoặc không có cơ sở để chứng minh là \textbf{đúng} sẽ được chấp nhận là \textbf{chưa biết}. 
\item[Ví dụ]
Xem xét một câu nói sau đây: "A là một công dân của nước Mỹ". Nếu có ai đó hỏi "A có phải là một công dân của Việt Nam hay không ?". Xét theo CWA, câu trả lời là \textit{không}, ngược lại với OWA thì câu trả lời là \textit{chưa biết}. 
\end{description}

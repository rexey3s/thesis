\chapter{Tổng quan đề tài}
\section{Lý do chọn đề tài}

\section{Khái quát nội dung đề tài}
\subsection{Ontology là gì ?}
Từ \textit{Ontology} được sử dụng trong ngành khoa học thông tin và ngành khoa học máy tính dùng để chỉ một mô hình hay hệ thống có khả năng biểu diễn hay thể hiện các đối tượng trong một lĩnh vực nào đó thông qua các định nghĩa về loại (phân lớp), thuộc tính và mối quan hệ giữa các đối tượng đó với nhau. Các mô hình ontologies thường được phát triển và sử dụng trong các lĩnh vực như trí tuệ nhân tạo, Web Ngữ nghĩa, phát triển phần mềm, công nghệ thông tin - sinh học, ... nhằm giảm sự phức tạp và tổ chức một khối lượng thông tin lớn và có nhiều ngữ nghĩa. Ngoài ra ontology còn có khả năng ứng dụng vào việc giải quyết vấn đề (problem solving).
\subsection{Ứng dụng vào đề tài}
Không giống với những ứng dụng phân loại đã có, trong nội dung khóa luận này chúng em đã chọn ngôn ngữ Ontology Web Languge và Semantic Web Rule Languge là những công nghệ nền tảng xây dựng tính năng phân loại không chỉ hàng hóa mà bất cứ đối tượng nào có thể định nghĩa bằng các ngôn ngữ trên. Chi tiết hơn về các công nghệ trên sẽ nằm trong nội dung báo cáo này. 
Trong quá trình nghiên cứu và tìm hiểu về các ngôn ngữ Ontology Web Language 2 (OWL 2), Semantic Web Rule Language (SWRL) vốn là nền tảng lý thuyết để xây dựng ứng dụng phân loại tự động, chúng em nhận thấy rằng để phát triển một mô hình Ontology dựa trên các loại ngôn ngữ nêu đòi hỏi phải xây dựng một trình biên tập các dữ liệu OWL 2, SWRL (chúng em sẽ trình bày rõ hơn trong nội dung của báo cáo). Trên thực tế, cũng đã có sẵn một số trình biên tập OWL 2 trên môi trường desktop như Protege, Swoop, hoặc trên môi trường web có WebProtege, tuy nhiên những trình biên tập này đã xuất hiện khá lâu, và một vài chương trình trong số đó đã ngừng phát triển hoặc không còn được cập nhật những thư viện mới nhất. Với những nguyên nhân kể trên, chúng em quyết định không sử dụng lại những trình biên tập đã cũ, mà sẽ tự xây dựng một ứng dụng biên tập và phát triển các OWL 2 Ontology, sau đó sử dụng các ontology này để trình bày quá trình phân loại tự động.
\section{Đối tượng và phạm vi nghiên cứu của đề tài}
Trong phạm vi của đề tài, chúng em chỉ xây dựng một ứng dụng biên tập ontology trong môi trường web và trình bày tính năng phân loại của ngôn ngữ OWL2, SWRL. Tính năng phân loại chỉ mang tính chất là một bản mẫu thực nghiệm, chứng minh các lý thuyết mà các công nghệ mới như OWL2 Ontology mang lại, chúng không có mục đích thay thế các hệ thống phân loại đang hoạt động và sử dụng rộng rãi.
Đề tài khóa luận đã làm những việc sau:
\begin{itemize}
\item Tìm hiểu về Semantic Web .
\item Tìm hiểu về ngôn ngữ Web Ontology Language (OWL) và Semantic Web Rule Language(SWRL) - hai ngôn ngữ này là nền tảng lý thuyết cho .
\item Tìm hiểu về tính nhất quán trong Ontology
\item Tìm hiểu về OWLAPI và SWRL API.
\item Tìm hiểu về Vaadin Framework.
\item Xây dựng thành công một trình biên tập OWL2 Ontology online với các tính năng biên tập ontology, suy luận reasoning, và hỗ trợ phân loại.
\end{itemize}

\subsection{Cấu trúc của khóa luận}
Khóa luận được chia thành các chương với nội dung như sau
\begin{enumerate}
\item Tổng quan đề tài
\item Cơ sở lý thuyết
\item Các công nghệ sử dụng
\item Xay dựng hệ thống phân loại
\item Kết luận 
\end{enumerate}

\chapter {Giới thiệu}
\section{Tên đề tài}

\section{Nội dung và giới hạn đề tài}
\subsection{Nội dung đề}
OWL ( Web Ontology Language ) là một dạng ngôn ngữ biểu diễn tri thức. Ngôn ngữ này thường được sử dụng phổ biến trong Semantic Web, và được trình bày dưới dạng RDF-XML. Ngày 27 tháng 10 năm 2009, tổ chức W3C (World Wide Web Consortium) cho công bố OWL 2, với trình chỉnh sửa Protégé và các bộ reasoner như Pellet, HermiT, v.v .
\\
OWL và Semantic Web hiện đang được nghiên cứu và phát triển, nhằm nhanh chóng đưa vào sử dụng, vì những lợi ích rất đáng kể, được xem là phiên bản web 3.0. Do đó, nhóm quyết định nghiên cứu về OWL, về phương thức hoạt động của các bộ reasoner, và đặc biệt là thiết kế một trình chỉnh sửa OWL trên web với giao diện thân thiện và dễ sử dụng, với các tính năng gần như đầy đủ so với chương trình Protégé. 
\\
Chắc chắn trong vài năm sắp tới, Semantic Web sẽ phát triển ngày càng lớn mạnh hơn, dần dần thay đổi phương thức tiếp cận và lưu trữ dữ liệu trên web. Vậy nên, việc tìm hiểu và nghiên cứu về ngôn ngữ OWL - một trong những thành phần quan trọng của Semantic Web, có thể coi như một bước “đón đầu công nghệ”, nhằm mục đích sẵn sàng thích nghi với sự chuyển biến không ngừng của thế giới công nghệ thông tin.
\\
Đề tài sẽ làm những việc sau:
\begin{enumerate}
\item Tìm hiểu về Semantic Web và Open World Assumption.
\item Tìm hiểu về ngôn ngữ Web Ontology Language (OWL) và Semantic Web Rule Language(SWRL).
\item Tìm hiểu về OWLAPI và SWRL API.
\item Tìm hiểu về nguyên lý hoạt động của OWL Reasoner (cụ thể là Pellet Reasoner).
\item Tìm hiểu về Vaadin Framework.
\item Sử dụng Vaadin Framework để xây dựng công cụ phục vụ phát triển Ontology trên web.
\item Giới thiệu những đặc điểm và tính năng nổi bật của phần mềm.
\item Kết luận và hướng phát triển nghiên cứu.
\end{enumerate}
\subsection{Giới hạn của đề tài}
Lĩnh vực Semantic Web là rất rộng lớn, nhóm chỉ tập trung nghiên cứu về OWL, về OWL API \cite{owlapi} và SWRL API \cite{swrlapi} để hiện thực chương trình chỉnh sửa, bên cạnh đó nghiên cứu hoạt động của Reasoner để hiện thực quá trình phân loại tự động, tìm hiểu về Vaadin Framework để xây dựng chương trình chỉnh sửa. Các vấn đề khác nằm ngoài tầm vóc của luận văn này.
\\
\subsection{Cấu trúc luận văn}
Luận văn được chia thành các chương như sau
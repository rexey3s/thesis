\chapter{Semantic Web và Ontology}
\section{Ý nghĩa của thuật ngữ Ontology trong ngành khoa học máy tính và khoa học thông tin}
\subsection{Nguồn gốc thuật ngữ ontology}
Nguồn gốc thuật ngữ ontology xuất hiện lần đầu tiên trong triết học \textsuperscript{*}, trích nguyên văn như sau:
\\
Ontology is the study of the nature of being, becoming, existence, or reality, as well as the basic categories of being and their relations.
\\
\textit{Tạm dịch:} Bản thể học là sự nghiên cứu về sự tồn tại, phát triển hoặc thực tại cũng như những những phân loại và quan hệ cơ bản giữa các bản thể.
\subsection{Ý nghĩa của ontology trong ngành khoa học máy tinh và ngành khoa học thông tin}
Một bản thể học được dùng để diễn giải tri thức thành một hệ thống các lớp nằm trong một lĩnh vực (domain) nhất định bằng cách sử dụng từ vựng chung để biểu thị các phân loại, các thuộc tính, các quan hệ tương quan giữa các lớp đó.
\\
Nhiều bộ bản thể học sẽ tạo thành các khuôn mẫu có cấu trúc để tổ chức thông tin thành một hệ thống có cấu trúc và được sử dụng trong các lĩnh vực như trí tuệ nhân tạo, hệ thống web ngữ nghĩa (Semantic Web)*, kỹ thuật hệ thống, kỹ thuật phần mềm, ngành thông tin sinh học dưới một dạng biểu diễn tri thức về một lĩnh vực hoặc một phần của nó. Việc tạo ra các bản thể học phục vụ cho một lĩnh vực có ý nghĩa nền tảng cho việc xây dựng nên \textit{enterprise architecture framework} \textsuperscript{**}  được mô tả là một hệ thống của hệ thống.
{\let\thefootnote\relax\footnotetext{*\textit{
			Ontology: http://en.wikipedia.org/wiki/Ontology}}
 \let\thefootnote\relax\footnotetext{**\textit{
			Enterprise architecture framework: http://en.wikipedia.org/wiki/Enterprise\_architecture\_framework}}
}
\section{Nguồn gốc ý tưởng về Semantic Web}

Ý tưởng hình thành nên Semantic Network Model bắt đầu từ những năm đầu 1960 bởi nhà khoa học \textit{Allan M.Collins}, nhà ngôn ngữ học \textit{M.Ross Quillian} và nhà tâm lý học \textit{Elizabeth}. W3C là tổ chức nghiên cứu và định nghĩa các thuật ngữ cho Semantic Web cũng như Ontology Web Language đứng đầu bới nhà khoa học \textit{Tim Berners-Lee}, người sáng lập World Wide Web.Ông đã định nghĩa về Semantic Web như sau: “Một trang web về dữ liệu mà máy móc có thể xử lý dữ liệu trên đó một cách trực tiếp và gián tiếp”. Mục đích chung của web ngữ nghĩa là giúp máy và các hệ thống điện toán có thể hiểu được những nội dung và ý nghĩa từ những nội dung mà con người vẫn hằng ngày tạo ra trên www, cuối cùng để máy có thể giúp hoặc làm thay con người những công việc như tìm kiếm, phân loại, xử lý thông tin trên web.